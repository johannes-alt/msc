\chapter{Measurements}
%Using the setup described previously, multiple measurements were performed. The noise coming from the electronic parts was measured, dark count spectra were measured, and also measurements with a LED as a light source were done. These measurements were first performed with one \ac{sipm} followed by measurements using multiple \ac{sipm}s connected parallel. For the measurement of the electronic noise, no high voltage was applied to the \ac{sipm}s. For the other measurements, the recommended operating voltage $U_\text{operating,rec.}=\SI{40.7}{\volt}$ was applied. 

After describing the different parts of the setup, now the measurements performed with these setups are presented. First, the setup presented in \autoref{fig:setup_inside_box} was assembled and measurements to find the correct delay value for the peak sensing cell were performed. Afterwards measurements in darkness without light exposure, followed by measurements with an LED as light source were performed.



\FloatBarrier
\section{Hold Scan}
One parameter that is important to set correctly when using the peak detector and crucial for using the SCA peak sensing option is the delay between the trigger and the hold signal. 
Set wrong while using the SCA, not the maximum of the shaper output amplitude but some amplitude before or after the maximum will be put out via the multiplexer. 
When using the peak sensing the exact delay value is not so important compared to the SCA. 
The delay must be longer than the time between the trigger and the maximum of the shaper output, otherwise the amplitude put to the multiplexer will be smaller than the true value. 
If the delay is to long it can happen, that another signal with higher amplitude happens during that delay and is measured instead of the signal triggering the event. 

In order to find the right delay time, a hold scan was performed. 
The hold scan uses the SCA peak sensing and measures with varying delay time. 
Starting with a \SI{1}{\nano\second} delay the delay was increased in \SI{1}{\nano\second} steps up to \SI{100}{\nano\second}. 
With each delay 1000 events were measured. 
The mean amplitude measured over all 1000 events is plotted over the delay.
A normalized waveform of a CRRC$^2$ shaper follows the function
\begin{align}
	S(t)&=\frac{1}{2}\cdot\left(\frac{t}{\tau}\right)^2\cdot\exp{-\frac{t}{\tau}}\label{eq:shaper_waveform_norm}
\end{align}
with the time $t$ and the time constant $\tau$ of the RC parts of the shaper.\cite{Kolanowsky}.
For the hold scan measurements, the amplitude $A$ of the signal and the offset $c$ of the waveform need to be included.
Also a time offset $t_\text{off}$ is necessary, since the travel time of the trigger and hold signal and the processing time of both increase the delay.
By including these parameters the \autoref{eq:shaper_waveform_norm} becomes
\begin{align}
	S(t)&=A\cdot\frac{1}{2}\cdot\left(\frac{t-t_\text{off}}{\tau}\right)^2\cdot\exp{-\frac{t-t_\text{off}}{\tau}}+c.\label{eq:shaper_waveform}
\end{align}
This function was fitted onto the hold scan measurement result to find the peaking time $t_\text{peak}$ at which the amplitude of the shaped signal is at its maximum.
With the \autoref{eq:shaper_waveform} the peaking time $t_\text{peak}$ and the time constant $\tau$ are related by
\begin{align}
	t_\text{peal}&=2\cdot\tau-t_\text{off}.
\end{align}
In \autoref{fig:plot_hold_scan_1} result of the hold scan done with one \ac{sipm} and the performed fit are shown.
The results of the hold scans performed with two to five \ac{sipm}s in parallel are plotted in the appendix. 
The fit parameter and the peaking time and theire uncertainties for the five hold scans are listed in \autoref{tab:hold_scan_fits}.
For one \ac{sipm} the peaking time is 
\begin{align}
	t_\text{peak,1}&=\SI{32.0(12)}{\nano\second}
\end{align}
and for multiple \ac{sipm}s in parallel this increases to
\begin{align}
	t_\text{peak,2}&=\SI{37.4(13)}{\nano\second}\\
	t_\text{peak,3}&=\SI{39.9(27)}{\nano\second}\\
	\text{and}\quad t_\text{peak,4}&=\SI{46.2(41)}{\nano\second}.
\end{align}
When using five \ac{sipm}s in parallel the peaking time is reduced to 
\begin{align}
	t_\text{peak,5}&=\SI{44.3(29)}{\nano\second}.
\end{align}
Since the hold scan done with four \ac{sipm}s lead to the fitted offset parameter to be much lower compared to the other measurements, it is likely, that for unknown reasons, the peaking time is higher than expected and the peaking time corresponding to five \ac{sipm}s is as expected.

\begin{table}
	\centering
	\caption[Hold scan fit parameters.]{The resulting parameters after fitting \autoref{eq:shaper_waveform} onto the hold scan measurements.}
	\setlength\extrarowheight{2.5pt}
	\begin{tabular}{l|
		S[table-format=4.0,detect-weight,mode=text] @{${}\pm{}$} S[table-format=3.0,detect-weight,mode=text]|
                S[table-format=3.0,detect-weight,mode=text] @{${}\pm{}$} S[table-format=3.0,detect-weight,mode=text]|
                S[table-format=2.1,detect-weight,mode=text] @{${}\pm{}$} S[table-format=1.1,detect-weight,mode=text]|
                S[table-format=2.1,detect-weight,mode=text] @{${}\pm{}$} S[table-format=1.1,detect-weight,mode=text]}
		\toprule
		number of \acp{sipm} & \multicolumn{2}{c}{$A$ / \si{\adcu}} & \multicolumn{2}{c}{\tau / \si{\nano\second}} & \multicolumn{2}{c}{$t_\text{off}$ / \si{\nano\second}} & \multicolumn{2}{c}{$c$ / \si{\adcu}}\\[2.5pt]\midrule
		1 & 
	\end{tabular}
\end{table}
The delays in the following measurements presented in this thesis were chosen based on this hold scan.
For measurements using the SCA, the measured peaking time, corresponding to the number of \ac{sipm}s used for the measurement, rounded to an integer value were used.
In case of the measurements performed with the peak detector the delay must be greater than the peaking time.
Therefore a delay of \SI{50}{\nano\second} was chosen regardless of the number of \ac{sipm}s.
A delay greater than \SI{50}{\nano\second} would increase the possibility of another event happening during the measurement time, thus measureing only the one with the higher amplitude.
For measurements with a light source, this would not harm the measurement, but the measurement of the dark count rate would be heavily influenced by this.
Since two dark counts happening in the time span of one event would lead to an undercount of dark counts and thus a smaller dark count rate.
But for the callibration measurement with dark counts the delay was set to \SI{255}{\nano\second}, the maximum value possible with the evaluation board.
Because of the higher probability of a dark count happening during each measurement time window, the number of counts in the noise peak decreases compared to the total number of counts and the relativ number of dark counts increases.

In the next section, the measurements to determine the dark count rate for \numrange{1}{5} \ac{sipm}s are presented.

\begin{figure}
	\centering
	\begin{subfigure}[t]{0.5\textwidth}
		\centering
		\includegraphics[width=1.\textwidth]{plots/hold_1.png}
		\caption{SiPM 21}
		\label{fig:plot_hold_scan_1}
	\end{subfigure}%
	\begin{subfigure}[t]{0.5\textwidth}
		\centering
		\includegraphics[width=1.\textwidth]{plots/hold_1.png}
		\caption{SiPMs 21, 22}
		\label{fig:plot_hold_scan_2}
	\end{subfigure}
	\begin{subfigure}[t]{0.5\textwidth}
		\centering
		\includegraphics[width=1.\textwidth]{plots/hold_1.png}
		\caption{SiPMs 21, 22, 23}
		\label{fig:plot_hold_scan_3}
	\end{subfigure}%
	\begin{subfigure}[t]{0.5\textwidth}
		\centering
		\includegraphics[width=1.\textwidth]{plots/hold_1.png}
		\caption{SiPMs 21, 22, 23, 24}
		\label{fig:plot_hold_scan_4}
	\end{subfigure}
	\begin{subfigure}[t]{0.5\textwidth}
		\centering
		\includegraphics[width=1.\textwidth]{plots/hold_1.png}
		\caption{SiPMs 21, 22, 23, 24, 25}
		\label{fig:plot_hold_scan_5}
	\end{subfigure}
	\caption[Hold scan with 1 \ac{sipm} and channel 1]{Hold scan performed with one \ac{sipm} connected to channel 1. The delay was changed from \SIrange{1}{100}{\nano\second} in \SI{1}{\nano\second} steps. With each delay 1000 measurements were made and the mean values for each delay are shown in this plot.}
	\label{fig:plot_hold_scan}
\end{figure}



\begin{table}
	\centering
	\caption[Hold scan results for 1 to 5 \ac{sipm}s in parallel.]{The results of the hold scans performed with one up to five \ac{sipm}s in parallel and the parameter obtained by fitting \autoref{eq:shaper_waveform} on the measurement data. $A$ being a scaling factor, $c$ is an offset to compensate the baseline, $t_\text{off}$ is a time offset and $\tau$ being the time constant of the shaper. The parameter $t_\text{peak}$ is the peaking time of the waveform, where it is at its maximum. The peaking time increases with the number of \ac{sipm}s up to four \ac{sipm}s, and decreases again with five \ac{sipm}s. These $t_\text{peak}$ rounded to an integer value were taken as delay for the measurements performed using the SCA. For the measurements with the peak detector the delay was set to \SI{50}{\nano\second}, if not stated otherwise, to ensure that the maximum of the shaped signal is in this delay time window.}
	\setlength\extrarowheight{2.5pt}
	\begin{tabularx}{\textwidth}{>{\centering}X
		S[table-format=4.0,detect-weight,mode=text] @{${}\pm{}$} S[table-format=3.0,detect-weight,mode=text]
		S[table-format=3.0,detect-weight,mode=text] @{${}\pm{}$} S[table-format=3.0,detect-weight,mode=text]
		S[table-format=2.1,detect-weight,mode=text] @{${}\pm{}$} S[table-format=1.1,detect-weight,mode=text]
		S[table-format=2.1,detect-weight,mode=text] @{${}\pm{}$} S[table-format=1.1,detect-weight,mode=text]
		S[table-format=2.1,detect-weight,mode=text] @{${}\pm{}$} S[table-format=1.1,detect-weight,mode=text]}
		\toprule
		N of \ac{sipm} & \multicolumn{2}{c}{$A$ / \si{\adcu}} & \multicolumn{2}{c}{$c$ / \si{\adcu}} & \multicolumn{2}{c}{$t_\text{off}$ / \si{\nano\second}} & \multicolumn{2}{c}{\tau / \si{\nano\second}} & \multicolumn{2}{c}{$t_\text{peak}$ / \si{\nano\second}} \\[2.5pt]\midrule
		1 & 456&9 & 916&3 & 23.6&8 & 27.8&4 & 32.0&1.2 \\[2.5pt]
		2 & 1380&40 & 882&10 & 20.1&9 & 28.7&5 & 37.4&1.3 \\[2.5pt]
		3 & 900&50 & 905&12 & 18.9&1.7 & 29&1 & 39.9&2.7 \\[2.5pt]
		4 & 6500&500 & 370&130 & 24&3 & 35.1&1.5 & 46.2&4.1 \\[2.5pt]
		5 & 2160&120 & 870&30 & 18.3&1.9 & 31&1 & 44.3&2.9\\[2.5pt]
		\bottomrule
	\end{tabularx}
\end{table}






\FloatBarrier
\section{Dark Count Rate}
In this section, a dark count measurement was performed in order to determine the \ac{dcr}. 
Knowing the \ac{dcr} is important to estimate the probability of an event in the low \si{\pe} range to be a dark count or caused by an incident photon.

For the measurement of the \ac{dcr}, the peak detector option of the peak sensing cell was chosen and the threshold bit value was set to 0 to be able to trigger also to noise.
Since with this threshold also the noise peak was measured and the Citiroc 1A evaluation board triggered with its maximal frequency, after one event, the next is triggered as soon as the evaluation board allows it.
Thus the maximum of the shaper amplitude is not necessary at the peaking time obtained by the hold scan. 
Therefore the delay can be chosen independently of the peaking time.
In general the longer the delay the higher the probability of multiple dark counts happen during one event, increasing the undercount of the \ac{dcr}.
So from the accuracy, setting the delay to \SI{1}{\nano\second}.
But this comes with the downside that, with an average number of approximatly 440 events per second, the measurement would need to run for a very long time to obtain a high enough statistic.
Therefor as a compromis between accuracy and efficiency the delay was set to \SI{50}{\nano\second}.


The \ac{dcr} can be calculated from the measured data using
\begin{align}
	\ac{dcr} &= \frac{N_\text{ev}(x>x_\text{th})}{t_\text{meas.}}\\
	&= \frac{N_\text{ev}(x>x_\text{th})}{N_\text{ev}\cdot t_\text{delay}}
\end{align}
with $N_\text{ev}(x>x_\text{thershold})$ being the number of events $N_\text{ev}$ with an amplitude $x$ greater than a threshold amplitude $x_\text{thershold}$ and $t_\text{meas.}$ being the measurement time.
Since the rate with which the Citiroc 1A evaluation board can measure events is around 440 events per second, the total number of events $N_\text{ev}$ times the delay $t_\text{delay}$ was used as total measurement time. 
This is possible, because the threshold slow control bit was set to 0, thus everything, even noise could trigger an event. 
Therefore the Citiroc 1A evaluation board recorded events with the maximal possible rate, which means, that the next event was triggered as soon as possible. 
Hence it is possible that the shaped signal of a dark count is already halfeway past when the triggering happens.
So the maximum of the shaper output signal can happen at any time during the time window between the triggering and the delaied hold signal making the total measurement time the total number of events times the delay.

The measurement to determine the dark count rate was performed for a single \ac{sipm} as well as for \numrange{2}{5} \ac{sipm}s read out in parallel.
The same \ac{sipm} group used for the hold scan measurements was chosen.
In \autoref{fig:plot_dcr_21} the histograms with the data measured using only \ac{sipm} 21 and five \asp{sipm} are shown.
The histograms for the measurements with two, three and four \asp{sipm} are shown in \autoref{fig:plot_dcr_2}, \autoref{fig:plot_dcr_3} and \autoref{fig:plot_dcr_4} in the appendix.
\autoref{rab:dc_rate} shows the results of the measurements with one to five \acp{sipm}.
In this table the total number of events, the number $N_\text{ev}(x>x_\text{th})$ of dark counts, the measurement time $t_\text{meas}$ and the dark count rate are listed.
Typically the dark count threshold $x_\text{th}$ is set to be \SI{0.5}{\pe}. 
Therefore it was chosen to be in middle between the noise peak and the \SI{1}{\pe}.
It can be calculated with
\begin{align}
	x_\text{th}&=\mu_1 - \frac{\mu_1-\mu_0}{2}
\end{align}
where $\mu_0$ and $\mu_1$ are the positions of the noise peak and \SI{1}{\pe} peak respectively.
The positions of the two peaks as well as the resulting thresholds for the five measurements are listed in \autoref{tab:dcr_threshold}.
For the meausrements with one, two and three \acp{sipm} the threshold is \SI{992}{\adcu} and for the measurements with four and five \acp{sipm} the threshold is \SI{993}{\adcu}.

\begin{table}
	\centering
	\caption[Dark count thresholds $x_\text{th}$]{The positions of the noise peaks and the \SI{1}{\pe} peak for the dark count rate measuremens using one to five \acp{sipm}. The dark count thresholds $x_\text{th}$ are the middle between the noise and \SI{1}{\pe} peaks rounded to an integer value.}
	\label{tab:dcr_threshold}
	\setlength\extrarowheight{2.5pt}
	\begin{tabularx}{1.\textwidth}{
		>{\centering\arraybackslash}X
		S[table-format=3.2,detect-weight,mode=text] @{${}\pm{}$} S[table-format=1.2,detect-weight,mode=text]
		S[table-format=4.2,detect-weight,mode=text] @{${}\pm{}$} S[table-format=1.2,detect-weight,mode=text]
		S[table-format=6.2,detect-weight,mode=text] @{${}\pm{}$} S[table-format=1.4,detect-weight,mode=text]
		>{\centering\arraybackslash}X}
		\toprule
		\# of \acp{sipm} & \multicolumn{2}{c}{$\mu_0$ / \si{\adcu}} & \multicolumn{2}{c}{$\mu_1$ / \si{\adcu}} & \multicolumn{2}{c}{$\mu_1-\frac{\mu_1-\mu_0}{2}$ / \si{\adcu}} & $x_\text{th}$ / \si{\adcu} \\[2.5pt]\midrule
		1 & 977.64 & 0.06 & 1005.97 & 0.18 & 991.80 & 0.10 & 992 \\[2.5pt]
		2 & 977.30 & 0.07 & 1006.82 & 0.17 & 992.06 & 0.09 & 992 \\[2.5pt]
		3 & 976.99 & 0.09 & 1007.98 & 0.22 & 992.48 & 0.12 & 992 \\[2.5pt]
		4 & 976.71 & 0.11 & 1009.40 & 0.40 & 993.06 & 0.20 & 993 \\[2.5pt]
		5 & 976.27 & 0.17 & 1010.48 & 0.22 & 993.37 & 0.14 & 993 \\[2.5pt]
		\bottomrule
	\end{tabularx}
\end{table}

\begin{figure}
	\centering
	\includegraphics[width=1.\textwidth]{plots/dc_gain_1.png}
	\caption[Dark count rate measurement histogram for SiPM 21]{Measurement data of the dark count rate measurement using \ac{sipm} 21. A gaussian function was fitted onto each peak. The mean values are $\mu_0=\SI{978.61(9)}{\adcu}$, $\mu_1=\SI{1007.8(1)}{\adcu}$, $\mu_2=\SI{1040.8(3)}{\adcu}$ and $\mu_3=\SI{1076.6(5)}{\adcu}$ for the \SI{0}{\pe}, \SI{1}{\pe}, \SI{2}{\pe} and \SI{3}{\pe} peaks respectively. The gray areas mark the sample range used to fit the gaussian on the corresponding peak.}
	\label{fig:plot_dcr_21}
\end{figure}



\begin{table}
	\caption[Dark count rate measurement results for \ac{SiPM} \numrange{21}{25}]{The results for the dark count measurements for the \ac{sipm}s \numrange{21}{25} read out indiviudally and in parallel. The first column shows the \ac{sipm} to which the measurement results in the corresponding row correspond. The second column shows the threshold $x_\text{th}$ which was used to devide the data in dark count events and noise events. The third and forth column show the total number of events and the number of dark count events based upon $x_\text{th}$. In column five is the total measurement time shown, calculated with the total event number and the \SI{50}{\nano\second} delay. In the last column the dark count rate is presented.}
	\label{tab:dc_rate_indiviual_sipms}
	\setlength\extrarowheight{2.5pt}
	\begin{tabularx}{1.\textwidth}{
		>{\raggedright\arraybackslash}X
		>{\centering\arraybackslash}X
		S[table-format=6.0,detect-weight,mode=text] @{${}\pm{}$} S[table-format=3.0,detect-weight,mode=text]
		>{\centering\arraybackslash}X
		S[table-format=7.0,detect-weight,mode=text] @{${}\pm{}$} S[table-format=4.0,detect-weight,mode=text]
		}
		\toprule
		\ac{sipm} &  $N_\text{ev}$ & \multicolumn{2}{c}{$N_\text{ev}(x>x_\text{th})$} & $t_\text{meas.}$ / \si{\second} & \multicolumn{2}{c}{DCR / events/\si{\second}} \\
		\midrule
		21                 & 806300 &  26392 & 160 & 0.040315 &  654600 & 4000 \\[2.5pt]
		21, 22             & 807500 &  52770 & 230 & 0.040375 & 1307000 & 5700 \\[2.5pt]
		21, 22, 23         & 807900 &  78504 & 280 & 0.040395 & 1943400 & 6900 \\[2.5pt]
		21, 22, 23, 24     & 808300 & 102534 & 320 & 0.040415 & 2537000 & 7900 \\[2.5pt]
		21, 22, 23, 24, 25 & 808200 & 128648 & 358 & 0.040410 & 3183600 & 8900 \\[2.5pt]
		\bottomrule
	\end{tabularx}
\end{table}

The \ac{dcr} for the single \ac{sipm} is around \SI{65e4}{\per\second}. 
Measuring dark counts with two \ac{sipm}s in parallel leads to an increase of the DCR of approximatly \SI{131e4}{\per\second}. 
Thus doubling the \ac{dcr} compared to one \ac{sipm}, as expected.
Adding a third \ac{sipm} in parallel results again in a DCR increase of around \SI{64e4}{\per\second}.
As expected, the \ac{dcr} increases by the amount measured with one \ac{sipm}.
By adding the fourth \ac{sipm}, the \ac{dcr} increases by \SI{60e4}{\per\second}. 
The smaller increase compared to the increases before could be explained by the undercounting due to the measurement methode with the peak detector and the \SI{50}{\nano\second} delay.
The higher the \ac{dcr}, the higher the probability of two darkcounts during the \SI{50}{\nano\second} delay resulting in the measurement of only one, therefore decreasing the measured \ac{dcr}.
Another possibility could be that the added \ac{sipm} produces less noise in form of dark counts.
With the fifth \ac{sipm} added in parallel, the DCR increases by approximatly \SI{65e4}{\per\second}.
That the \ac{dcr} increase is simillar to the \ac{dcr} of a single \ac{sipm} indicates, that the lower increas due to the forth \ac{sipm} is due to it's lower noise.
It also leads to the assumption, that the undercount is not high enough to affect the measurement siginificantly.

These measurement results show, that the Citiroc 1A \ac{asic} as well as the Citiroc 1A evaluation board are suitable for measureing the dark count rate of up to five \acp{sipm} read out in parallel.
Since the \ac{dcr} depends on the \ac{sipm} model, temperature and over voltage, the conclusion does not necessary holds, when changing these parameters.

In the next part of this chapter, \ac{dc} measurements are used to calibrate the \acp{sipm}.


\FloatBarrier
\section{Gain Determination with Dark Counts}
With the same setup previously used to determine the \ac{dcr} also the gain of the \acp{sipm} can be measured.
Being able to determine the gain by measureing only \acp{dc} is important for the \ac{sbt}.
Otherwise a lightsource would be needed to performe the calibration of every \ac{sipm} array.
This would mean either every \ac{pcb} with \acp{sipm} would need to house a LED or one would need a radioactive source outside the \ac{sbt} cells and measure the scintillation light. 
Which is a far bigger effort compared to a \ac{dc} measurement.

For this measurement, the same slow control values used in the \ac{dcr} measurement were used, only the delay was changed.
To measure more \ac{dc} events compared to noise events, the delay was set to \SI{255}{\nano\second}.
This calibration measurement was performed for one to five \acp{sipm} in parallel.
The duration of each measurement was set to \SI{20}{\minutes}.


In this section, the results of one group of five \acp{sipm}, the \acp{sipm} 21 to 25, are presented.
The measurement data taken with the single \ac{sipm} 21 is presented in the histogram in \autoref{fig:dc_gain_21}.
A single gaussian function
\begin{align}
	Gaussian(x) &= A\cdot\exp{-\frac{(x-\mu)^2}{2\sigma^2}}
\end{align}
with the amplitude $A$, the mean $\mu$ and the variance $\sigma^2$ was fitted onto each individual peak and the results as well as the sum of the gaussians are plotted in \autoref{fig:dc_gain_21}.
The range used for each fit is marked in gray.
By taking the differences between the mean peak positions, the gain can be calculated.
Because the peak detector always remembers the maximum of the shaper output, the noise peak is not exactly where a \SI{0}{\pe} signal would be expected to be.
Therefore the noise peak was excluded from this calculation.
For the calibration of the \acp{sipm} the position of one photoelectron peak has to be known besides the gain. 
Since the noise peak can very from where a \SI{0}{\pe} peak would be expected, the \SI{1}{\pe} peak was chosen.
The position of the \SI{1}{\pe} peak as well as the calculated gain are listed in \autoref{tab:dc_rate}.


\begin{landscape}
\begin{table}
	\centering
	\caption[Dark count calibration results for \ac{SiPM}s \numrange{36}{40}]{The calibration results for the \ac{sipm}s \numrange{36}{40} read out indiviudally and in parallel. The \ac{sipm}s are listed with the position of the 1 PE peak and the gain.}
	\label{tab:dc_rate}
	\setlength\extrarowheight{2.5pt}
	\begin{tabularx}{1.\linewidth}{>{\raggedright\arraybackslash}X
		%S[table-format=3.2,detect-weight,mode=text] @{${}\pm{}$} S[table-format=1.2,detect-weight,mode=text]|
		S[table-format=4.2,detect-weight,mode=text] @{${}\pm{}$} S[table-format=1.2,detect-weight,mode=text]
		S[table-format=4.1,detect-weight,mode=text] @{${}\pm{}$} S[table-format=1.1,detect-weight,mode=text]
		S[table-format=4.1,detect-weight,mode=text] @{${}\pm{}$} S[table-format=1.1,detect-weight,mode=text]
		S[table-format=4.1,detect-weight,mode=text] @{${}\pm{}$} S[table-format=1.1,detect-weight,mode=text]
		S[table-format=4.1,detect-weight,mode=text] @{${}\pm{}$} S[table-format=1.1,detect-weight,mode=text]
		S[table-format=6.1,detect-weight,mode=text] @{${}\pm{}$} S[table-format=1.7,detect-weight,mode=text]}\toprule
		\ac{sipm} & \multicolumn{10}{c}{peak position / [ADCu]} & \multicolumn{2}{c}{gain / [ADCu/PE]} \\[2.5pt]
		  & \multicolumn{2}{c}{\SI{1}{pe}} & \multicolumn{2}{c}{\SI{2}{pe}} & \multicolumn{2}{c}{\SI{3}{pe}} & \multicolumn{2}{c}{\SI{4}{pe}} & \multicolumn{2}{c}{\SI{5}{pe}} & \multicolumn{2}{c}{} \\[2.5pt]\midrule
		21                 & 1007.87 & 0.11 & 1040.8 & 0.3 & 1078.0 & 0.5 & \multicolumn{2}{c}{\mbox{}} & \multicolumn{2}{c}{\mbox{}} & 34.4 & 0.3 \\[2.5pt]
		21, 22             & 1008.32 & 0.12 & 1042.0 & 0.2 & 1077.9 & 0.5 & 1115.0 & 1.3 & \multicolumn{2}{c}{\mbox{}} & 35.5 & 0.4 \\[2.5pt]
		21, 22, 23         & 1009.95 & 0.12 & 1045.5 & 0.2 & 1083.5 & 0.5 & 1119.9 & 1.1 & \multicolumn{2}{c}{\mbox{}} & 36.7 & 0.4 \\[2.5pt]
		21, 22, 23, 24     & 1011.40 & 0.16 & 1048.3 & 0.3 & 1087.6 & 0.7 & 1127.4 & 2.6 & 1164.5 & 1.7 & 38.3 & 0.4 \\[2.5pt]
		21, 22, 23, 24, 25 & 1012.93 & 0.17 & 1051.7 & 0.3 & 1092.8 & 1.1 & 1134.0 & 1.5 & \multicolumn{2}{c}{\mbox{}} & 40.4 & 0.5 \\[2.5pt]
		\bottomrule
	\end{tabularx}
\end{table}
\end{landscape}



%\multicolumn{2}{c}{noise} & 
%978.61 & 0.09 & 
%978.32 & 0.10 & 
%978.12 & 0.12 & 
%977.84 & 0.15 & 
%977.80 & 0.16 & 



\FloatBarrier
\section{Measurements with a Light Source}
After the dark count measurements were performed, the fiber coming from the LED was mounted on the optical rail facing the \ac{sipm} array.
To garantee an even light exposure for all \ac{sipm}s the diffusor was placed between the fiber and the \ac{sipm} array.
This addition to the setup is depicted in \autoref{fig:LED_setup_pic}.

The \SI{460}{\nano\meter} light emitted by the LED was guided through a optical fiber into the dark measurement box.
There it was placed, pointing towards the \ac{sipm} array, behind a diffusor which diffused the light evenly in a circle with an \SI{50}{\degree} opening angle.
The LED was powered by the \textit{Tektronix AFG 3252} arbitrary waveform generator.
In order to create short light signals, a pulse waveform with a \SI{10}{\nano\second} width. 
Both rise and fall time of the pulse were set to the minimal possible value of \SI{2.5}{\nano\second}. 
The baseline was set to \SI{0}{\volt} and the amplitude was changed between measurements, to measure approximatly the same number range of photons in each measurement.
Otherwise the measurements with more than one \ac{sipm} would measure accordingly higher mean photon numbers.
Since the resolution of the \si{\pe} peaks also depend on the number of \si{\pe} and the linearity of the different electric components,e.g. amplifier, shaper, \ac{adc}, which can depend on the amplitude of the signal, a great difference in the mean photon number between measurements would influence the comparison of the measurement results.

The slow control settings were mostly left unchanged from the \ac{dc} measurements.
The only slow control difference was, that the threshold for both charge and time trigger was set to the slow control value 300.
By setting the threshold to 300 the noise peak and the first few \si{\pe} peaks were cut off. 
Otherwise these peaks would dominate the measurement and thus the measurement time would need to be increased heavily to obtain the same number of events caused by the LED light.


First the measurements with single \ac{sipm}s were performed. 
The duration of the measurement was set to \SI{20}{\minute}.
In \autoref{fig:LED_measurement_sipm21} the histogram for the measurement with \ac{sipm} 21 is shown.
With Python the number $n_\text{peaks}$ of recognisable peaks and there rough positions in \si{\adcu} were determined.
A sum of gaussian functions
\begin{align}
	G_\text{sum}(x) &= \sum_{i}^{n_\text{peaks}}A_i\cdot\exp{-\frac{(x-\mu_i)^2}{2\sigma_i^2}}
\end{align}
with the amplitudes $A_i$, mean values $\mu_i$ and the variances $\sigma_i^2$ was fitted onto the data.
For the $\mu_i$ boundaries were set to ensure that the peaks were fitted at roughly the expected \si{\adcu} range and to exclude the possibility that the fit function would prefer a nonsensical fit.
The fit parameter and the corresponding uncertainties are listed in the appendix in \autoref{tab:LED_sipm21_fitpara}.\textbf{TODO}
The result of this fit is also plotted in \autoref{fig:LED_measurement_sipm21}.

In order to determine the gain of this \ac{sipm} with the used setup, the mean values of the peaks were plotted in \autoref{fig:LED_sipm21_gain} over the \si{\pe} number and a linear fit 
\begin{align}
	f(x)&=a\cdot x +c 
\end{align}
was performed.
The slope $a$ of the fitted function correspondes to the gain of the \ac{sipm}.
With the knowledge of the approximate \SI{0}{\pe} position from the \ac{dc} measurement, the linear fit was used to extrapolate the expected position and therefore determine to how many \si{\pe} the peaks correspond.
Both values are listed for one to five \acp{sipm} listed in \autoref{fig:LED_gain_pd_result}.

This gain can be compared to the gain determined via the dark count measurements.

\begin{table}
	\centering
	\caption[Gain and \SI{0}{\pe} position for \ac{SiPM}s \numrange{21}{25}]{The results of the calibration performed with one to five \ac{sipm}s in parallel. The \acp{sipm} \numrange{21}{25} were chosen. The \ac{sipm}s are listed with the position of the \SI{0}{\pe} peak and the gain.}
	\label{tab:dc_rate}
	\setlength\extrarowheight{2.5pt}
	\begin{tabularx}{1.\linewidth}{>{\raggedright\arraybackslash}X
		S[table-format=2.2,detect-weight,mode=text] @{${}\pm{}$} S[table-format=1.5,detect-weight,mode=text]
		S[table-format=6.1,detect-weight,mode=text] @{${}\pm{}$} S[table-format=1.7,detect-weight,mode=text]}\toprule
		\ac{sipm} & \multicolumn{2}{c}{gain / \si{\adcu\per\pe}} & \multicolumn{2}{c}{\SI{0}{\pe} position / \si{\adcu}} \\[2.5pt]\midrule
		21                 & 35.26 & 0.03 & 953.0 & 0.4 \\[2.5pt]
		21, 22             & 38.68 & 0.06 & 960.8 & 0.7 \\[2.5pt]
		21, 22, 23         & 40.32 & 0.04 & 956.4 & 0.5 \\[2.5pt]
		21, 22, 23, 24     & 41.48 & 0.06 & 953.1 & 0.7 \\[2.5pt]
		21, 22, 23, 24, 25 & 43.09 & 0.05 & 950.5 & 0.7 \\[2.5pt]
		\bottomrule
	\end{tabularx}
\end{table}

It is clear, that the gain increases with the number of \acp{sipm} connected in parallel, at least up to five \acp{sipm}.
This behavior is the exact oposite of what is expected from simulations \cite{bsc_jonathan}.
From the simulation, the gain should decrease with increasing number of \acp{sipm} read out in parallel.
The reason for this deviation from the simulations is unkown.
A possiblility could be effects caused by the \ac{pcb} on which the \acp{sipm} are soldered on.
For future measurements, for example, the capacitors between the high voltage and ground could be exchanged by other capacitors.






%\FloatBarrier
%\subsection{Resolution}
%Another important quantity is the resolution of a detector.
%For gaussian funcions, the resolution is defined by
%\begin{align}
%	resolution &= \frac{\sigma}{\mu}.
%\end{align}
%For the measurements done using the LED, the resolution was calculated for every peak.
%Since the Citiroc 1A evaluation board does have a channel dependend baseline at around \SI{960}{\adcu} and not at \SI{0}{\adcu}, the $\mu$ of the 0 PE peak was substracted.
%The results are plotted in \autoref{fig:LED_resolution} and listed in \autoref{tab:LED_resolution}.

















