\chapter{Results of the \ac{daq} tests}


\section{Input offset voltage}
The first measurement is to find out the input offset voltage of the \ac{emusic} for different \ac{dac} values.
This needs to be done to find out the exact overvoltage of the \acp{sipm}.
In order to perform this measurement, the setup in \autoref{fig:input_offset_setup} was assembled.
The \ac{emusic} board was powered by the \SI{8}{\volt} power supply and the XXXXXXXX power supply was used to supply the high voltage.
It was set to \SI{16}{\volt} since it was unimportant for the high voltage to be over the breakdown voltage of the \acp{sipm}.
It should only be over the maximum input offset the \ac{emusic} can generate for the resulting voltage to be applied in reverse direction to the \acp{sipm}.
For this measurement the \ac{emusic}s input \ac{dac} settings, which can range from 0 to 511, were set to 0.
Then the voltages between the negative high voltage pole and the voltage on the cathode of the \acp{sipm} were measured.
This measurement was done for one \ac{sipm} of each \ac{sipm} group.
The chosen \ac{sipm} were 1, 6, 11, 16, 21, 26, 31 and 36.
After measureing the voltages, the \ac{dac} setting was increased in steps of 50 \ac{dac}u.
For each setting the eight voltages were measured again.

The measurement for 




\section{Pole-zero cancellation}

In this section pole-zero cancellation shaper and its effect with different settings tested.
The tests were done using the setup described in \autoref{sec:setup}.
For the amplification the \ac{emusic} board 2 was chosen and the digitization was done by a GANDALF module.
The \ac{emusic} settings of one of these measurements are shown in \autoref{} in the appendix.
For the other measurements, the only thing changing in the settings are the pole-zero settings.
Each measurement includes XXXXX events, for which the mean value is shown in the plots below.
The values for the amplitude and \ac{fwhm} were calculated for each individual waveform and theire mean values are presented here for the different measurements.

First the pole-zero cancellation was disabled to perform measurements for a reference amplitude and \ac{fwhm}.
In \autoref{fig:pz_no_pz} the mean of all waveforms, measured in channel 0, is shown.
The mean amplitude is 
\begin{align}
    V_{amp} &= \SI{1(1)}{\milli\volt}
\end{align}
and the \ac{fwhm} is
\begin{align}
    t_\text{FWHM} &= \SI{1(1)}{\nano\second}.
\end{align}

Next the measurement with enabled pole-zero cancellation and with fixed settings for its capacitor and variing resistor values.
The resistor settings were changed to all possible values, from 0 to 7, and the capacitor setting was kept at 31.
\autoref{fig:pz_resistor} presents the mean waveforms for the eight measurements.
The determined amplitudes and \ac{fwhm} and the coresponding decrease in respect to the values without pole-zero cancellation are listed in \autoref{tab:pz_resistor}.
Of all measurements the one with the resistor setting 0 had the highest mean amplitude with 
\begin{align}
    V_\text{amp,R=0} &= \SI{1(1)}{\milli\volt}
\end{align}
which correspondes to a amplitude attenuation of \SI{1(1)}{\percent}.
With these settings a \ac{fwhm} of
\begin{align}
    t_\text{FWHM,R=0} &= \SI{1(1)}{\nano\second}
\end{align}
was measured, which is \SI{1(1)}{\percent} of the \ac{fwhm} measured without pole-zero cancellation.
\begin{figure}
	\centering
	\includegraphics[width=1.\textwidth]{pictures/pz_resistor.png}
	\caption[Mean waveform for different pz-cancellation resistor values.]{Mean waveform for different pz-cancellation resistor values.}
	\label{fig:pz_resistor}
\end{figure}

The measurements with different capacitor settings were performed with the resistor setting of 0.
For the different measurements, the capacitor settings were changed to all 32 possible values, from 0 to 31.
In \autoref{tab:pz_capacitor} the mean values for the maximum amplitude and the \ac{fwhm} as well as the decrease compared to the values with disabled pole-zero cancellation are listed.
The mean waveforms for the measurements with the capacitor settings 0, 5, 10, 15, 20, 25, and 31 are shown in \autoref{fig:pz_capacitor}.
\begin{figure}
	\centering
	\includegraphics[width=1.\textwidth]{pictures/pz_capacitor.png}
	\caption[Mean waveform for different pz-cancellation capacitor values.]{Mean waveform for different pz-cancellation capacitor values.}
	\label{fig:pz_capacitor}
\end{figure}





\section{High and low transimpedance and lower attenuation}



\section{Dark count measurements}

