\chapter{Results of the \ac{daq} tests}


\section{GANDALF Clock Frequency and ADC Test}
Before using the \ac{gandalf} modules in the \ac{daq}, they need to be tested.
For this to things are relevant.
One of these is the sample frequency, if it correspondes to the theoretical sample frequency.
The other is the test of the \acp{adc} to see if they have dead bits or other malfunctions.
To perform this test a \SI{150}{\mega\hertz} sin voltage signal was generated with an AWG and a \SI{150}{\mega\hertz} filter by XXXXXXX was used.
This filter ensures that the sin signal is clean and no unwanted frequencies are present.
The clean sin signal was than connected to the inputs of the \acp{gandalf}, one after another.
For each input \num{1000}1000 measurements were done each with XXXXX samples.
For each channel a \ac{fft} was done for one waveform using the python3 functions \textit{scipy.fft.rfft} and \textit{scipy.fft.rfftfreq} to find the frequencies in the signal.
\autoref{fig:gandalf_input_g23_ch0} shows the plot with the \ac{fft} for channel 0 of the \ac{amc} 46 used with the \ac{gandalf} 24.
In the plot the red line marks the \SI{150}{\mega\hertz} of the sin signal and the green line marks the peak frequency of the measured signal.
The measured values and the corresponding uncertainties are listed in \autoref{tab:gandalf_sin_150}.




To test if the bits in the \acp{adc} are working, the amplitudes of all samples of every waveforms in a measurement were put into histogram.





\section{Input Offset Voltage}
The first measurement with the \ac{emusic} board is to find out the input offset voltage of the \ac{emusic} for different \ac{dac} values.
This needs to be done to find out the exact overvoltage of the \acp{sipm}, which influences among other things the gain of the \acp{sipm}.
In order to perform this measurement, the setup in \autoref{fig:input_offset_setup} was assembled.
The \ac{emusic} board was powered by the \SI{8}{\volt} power supply and the XXXXXXXX power supply was used to supply the high voltage.
It was set to \SI{4.7}{\volt} since it was not important for the high voltage to be over the breakdown voltage of the \acp{sipm}.
It should only be over the maximum input offset the \ac{emusic} can generate for the resulting voltage to be applied in reverse direction to the \acp{sipm}.
For this measurement the \ac{emusic}s input \ac{dac} settings, which can range from 0 to 511, were set to 0.
Then the voltages between the negative high voltage pole and the voltage on the cathode of the \acp{sipm} were measured.
As measurement point for the cathode voltage, the \SI{0}{\ohm} resistor placed on the back of the \ac{sipm} board was chosen.
It is shown in \autoref{fig:meas_input_resistor}.
This measurement was done for one \ac{sipm} of each \ac{sipm} group.
The chosen \ac{sipm} were 1, 6, 11, 16, 21, 26, 31 and 36.
After measureing the voltages, the \ac{dac} setting was increased in steps of \SI{50}{\dacu} until \SI{500}{\dacu} and at each step the measurement was repeated.
In the following first the measurement results of the individual channels over all tested \ac{dac} settings are presented.
Afterwards, the input offset voltages of the different channels for the same setting are compared.

As an example, the measurements of all eleven tested \ac{dac} settings done with the \ac{emusic} board 2 and channel 0 are shown in \autoref{fig:input_offset_b2_ch0}.
In the upper part of the plot, the measured voltages are plotted and for the measurements with a \ac{dac} setting between \SI{100}{\dacu} and \SI{100}{\dacu} a linear fit was performed.
The first two measured voltages were not included, since they visibly do not follow the linear trend.
For all channels the first to measured values are at around \SI{1530}{\milli\volt} which indicates that in that \si{\dacu} range the different settings do not change the offset voltage.
Depending on the \ac{emusic} board and the channel also the last one or two measured voltages were excluded from the fit since they also do not follow the linear trend and increase to around \SI{940}{\milli\volt} instead of decreasing further.
The resulting slope and offset of the linear fit are
\begin{align}
    V_\text{offset, fit} &= \SI{-3.43(5)}{\milli\volt\per\dacu}\cdot x+\SI{1828(11)}{\milli\volt}
\end{align}
where x is the setting of the \ac{dac} in \si{\dacu}.
The bottom of the plot shows the residual plot where the difference between the linear fit and the measured values is shown.
The range on the y-axis is fixed to \SIrange{-20}{20}{\milli\volt}.
From that one can see, that should be a precission of \SI{+-20}{\milli\volt} sufficient for ones used, one can use the results from the linear fit to adjust the overvoltage.
In case one wants a precission in the single \si{\milli\volt} range, this is not suitable anymore.
Than the \ac{dac} needs to be adjusted while measureing the voltage on the inputs of the \ac{emusic} \ac{asic}.
In \autoref{tab:input_offset_linear_fit} the fit results for the measurements with the \ac{emusic} boards 2 and 6 are listed.
The \ac{dac} voltages of the channels differ to other channels on the same board as well as to the channels on the other board.
Therefore the measurement of the input voltage should be done for every \ac{emusic} board.
\begin{figure}
	\centering
	\includegraphics[width=1.\textwidth]{pictures/input_offset_board_2_channel_0}
	\caption[Input offset measurement for eMUSIC board 2 channel 1]{Input offset measurement for the channel 0 of the \ac{emusic} board 2. The input voltage was measured for input \ac{dac} settings from \SIrange{0}{500}{\dacu} in \SI{50}{\dacu} steps. A linear fit was performed for the measurements with \ac{dac} settings between \SI{100}{\dacu} and \SI{450}{\dacu}. The other measured voltages were excluded form the fit since they do not follow the linear trend. Below is the residual plot with a fixed y-axis window from \SI{-20}{\milli\volt} to \SI{-20}{\milli\volt}.}
	\label{fig:input_offset_b2_ch0}
\end{figure}
\begin{table}
	\centering
	\caption[todo]{todo}
	\label{tab:input_offset_linear_fit}
	\begin{tabular}{|c|c|c|c|}
		a & a & A & a
	\end{tabular}
\end{table}

To compare the differences between channels with the same \ac{dac} settings, for three different settings the input voltage was plotted for all eight channels in \autoref{fig:input_offset_b2_dac}, \autoref{fig:input_offset_b2_dac}, and \autoref{fig:input_offset_b2_dac}.
For settings at \SI{50}{\dacu} and below, the input offset voltage is pretty equal between the channels and only differs in the single \si{\milli\volt} range around \SI{0}{\milli\volt}.
A similiar behavior is seen for \ac{dac} settings at and above \SI{500}{\dacu}, for which the input voltage is around \SI{940}{\milli\volt} and the differences between the channels is also in the single \si{milli\volt} range.
But in the \si{dacu} range where the linear progression can be seen, the variations between the channels is larged, in some cases over \SI{70}{\milli\volt}.
This confirms, that the measurement of the input voltage needs to be done for all \ac{emusic} boards and all channels.
\begin{figure}
	\centering
	\begin{subfigure}[b]{1.\textwidth}
		\centering
		\includegraphics[width=.5\textwidth]{pictures/input_offset_board_2_channel_0.png}
		\caption{}
		\label{fig:input_offset_b2_dac50}
	\end{subfigure}
	
	\begin{subfigure}[b]{1.\textwidth}
		\centering
		\includegraphics[width=.5\textwidth]{pictures/input_offset_board_2_channel_0.png}
		\caption{}
		\label{fig:input_offset_b2_dac50}
	\end{subfigure}

	\begin{subfigure}[b]{1.\textwidth}
		\centering
		\includegraphics[width=.5\textwidth]{pictures/input_offset_board_2_channel_0.png}
		\caption{}
		\label{fig:input_offset_b2_dac50}
	\end{subfigure}
	\caption[todo]{todo}
	\label{fig:input_offset_b2_dac}
\end{figure}

After these measurements were done, for the \ac{emusic} boards 2 and 6 the settings for which all channels have \SI{1}{\volt} offset were determined.
The resulting \ac{dac} settings and the corresponding input voltages are listed in \autoref{tab:input_offset_equalized}.
With these settings the measurement of the pole-zero cancellation and the low and high trans-impedance and pole-zero attenuation were performed.
Which are presented in the following sections.
\begin{table}
	\centering
	\caption[todo]{todo}
	\label{tab:input_offset_equalized}
	\begin{tabular}{|c|c|c|c|}
	    \ac{emusic} board & channel & \ac{dac} settings / \si{\dacu} & input offset voltage / \si{\milli\volt} \\\hline
	\end{tabular}
\end{table}

\section{Pole-Zero Cancellation}

In this section pole-zero cancellation shaper and its effect with different settings tested.
The tests were done using the setup described in \autoref{sec:setup}.
For the amplification the \ac{emusic} board 2 was chosen and the digitization was done by a GANDALF module.
The \ac{emusic} settings of one of these measurements are shown in \autoref{} in the appendix.
For the other measurements, the only thing changing in the settings are the pole-zero settings.
Each measurement includes XXXXX events, for which the mean value is shown in the plots below.
The values for the amplitude and \ac{fwhm} were calculated for each individual waveform and theire mean values are presented here for the different measurements.

First the pole-zero cancellation was disabled to perform measurements for a reference amplitude and \ac{fwhm}.
In \autoref{fig:pz_no_pz} the mean of all waveforms, measured in channel 0, is shown.
The mean amplitude is 
\begin{align}
    V_{amp} &= \SI{1(1)}{\milli\volt}
\end{align}
and the \ac{fwhm} is
\begin{align}
    t_\text{FWHM} &= \SI{1(1)}{\nano\second}.
\end{align}

Next the measurement with enabled pole-zero cancellation and with fixed settings for its capacitor and variing resistor values.
The resistor settings were changed to all possible values, from 0 to 7, and the capacitor setting was kept at 31.
\autoref{fig:pz_resistor} presents the mean waveforms for the eight measurements.
The determined amplitudes and \ac{fwhm} and the coresponding decrease in respect to the values without pole-zero cancellation are listed in \autoref{tab:pz_resistor}.
Of all measurements the one with the resistor setting 0 had the highest mean amplitude with 
\begin{align}
    V_\text{amp,R=0} &= \SI{1(1)}{\milli\volt}
\end{align}
which correspondes to a amplitude attenuation of \SI{1(1)}{\percent}.
With these settings a \ac{fwhm} of
\begin{align}
    t_\text{FWHM,R=0} &= \SI{1(1)}{\nano\second}
\end{align}
was measured, which is \SI{1(1)}{\percent} of the \ac{fwhm} measured without pole-zero cancellation.
\begin{figure}
	\centering
	\includegraphics[width=1.\textwidth]{pictures/pz_resistor.png}
	\caption[Mean waveform for different pz-cancellation resistor values.]{Mean waveform for different pz-cancellation resistor values.}
	\label{fig:pz_resistor}
\end{figure}

The measurements with different capacitor settings were performed with the resistor setting of 0.
For the different measurements, the capacitor settings were changed to all 32 possible values, from 0 to 31.
In \autoref{tab:pz_capacitor} the mean values for the maximum amplitude and the \ac{fwhm} as well as the decrease compared to the values with disabled pole-zero cancellation are listed.
The mean waveforms for the measurements with the capacitor settings 0, 5, 10, 15, 20, 25, and 31 are shown in \autoref{fig:pz_capacitor}.
\begin{figure}
	\centering
	\includegraphics[width=1.\textwidth]{pictures/pz_capacitor.png}
	\caption[Mean waveform for different pz-cancellation capacitor values.]{Mean waveform for different pz-cancellation capacitor values.}
	\label{fig:pz_capacitor}
\end{figure}





\section{High and Low Transimpedance and Lower Attenuation}
After investigating the effects of the pole-zero shaper the influences between the normal and the low pole-zero attenuation settings are examined.
Also the high and low transimpedance settings are investigated.

First the pole-zero attenuation was looked at.
Therefore another meausrement without the lower attenuation setting was performed.
The pole-zero cancellation resistor setting was set to 3 and the capacitor setting was set to 31. 


After the pole-zero attenuation, the high and low transimpedance settings were investigated.
Therefore two measurements without pole-zero cancellation were taken with and without the high transimpedance.
Both measurements consist of around \num{30000} waveforms.
A mean waveform was calculated for both measurements and plotted in \autoref{fig:low_imp_wf}.
The mean amplitude of the events with high transimpedance is
\begin{align}
	V_\text{high imp} &= \SI{530(40)}{\milli\volt}
\end{align}
and of the events with low transimpedance the mean amplitude is
\begin{align}
	V_\text{low imp} &= \SI{175(13)}{\milli\volt}.
\end{align}
This results in a decrease by the factor \num{0.33(4)} if the low transimpedance is used.
Doing the same measurements using the pole-zero cancellation with both the resistor and capacitor settings set to 0 resulted in mean amplitudes of
\begin{align}
	V_\text{high imp} &= \SI{200(16)}{\milli\volt}\\
	\text{and}\quad V_\text{low imp} &= \SI{72(5)}{\milli\volt}
\end{align}
for high and low transimpedance respectively.
The signal amplitude decreases by the factor \num{0.36(4)}, which is in agreement with the value determined without pole-zero cancellation.
These measurements were repeated with the settings 3 and 31 for the pole-zero resistor and capacitor, respectively.
The mean amplitudes for these measurements are
\begin{align}
	V_\text{high imp} &= \SI{260(20)}{\milli\volt}\\
	\text{and}\quad V_\text{low imp} &= \SI{91(7)}{\milli\volt}
\end{align}
and the factor by which the signal amplitude decreases is \num{0.35(4)}, which is also in agreement with the results of the other two measurements.
The plots of the mean waveforms for the measurements with pole-zero cancellation are shown in the appendix in \autoref{} and \autoref{}.
\begin{figure}
	\centering
	\includegraphics[width=1.\textwidth]{pictures/low_imp_mean_wf.png}
	\caption[todo]{todo}
	\label{fig:low_imp_wf}
\end{figure}

After the, for this work, most important features of the \ac{emusic} are investigated, a \ac{emusic} board and a Hamamatsu \ac{sipm} board were used to measure dark counts.
The goal of the dark count measurement was to test if dark counts could be used for calibration.

\section{Dark Count Measurements}

