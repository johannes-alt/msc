\chapter{Summary and Outlook}
In this thesis a \ac{daq} for the readout of \acp{wom} was assembled and tested.
It consists of \ac{emusic} boards with \ac{emusic} chips for signal amplification and shaping and \ac{gandalf} moduls for digitization.
In the future it is supposed to be used for the long term test of the so called \qq{One Cell Prototype} in Freiburg.

A external clock was prepared for the \acp{gandalf} and tested for proper functionality.
This test was done by generating a pure \SI{150}{\mega\hertz} sine waveform with a arbitrary waveform generator and a \SI{150}{\mega\hertz} frequency filter.
The peak frequency of the digitized frequency was determined with a \ac{fft} is \SI{149.4(10)}{\mega\hertz} which is in agreement with the theoretical frequency of the sine waveform.
Therefore it can be concluded that the \acp{gandalf} sampling with the correct frequency of \SI{880}{\mega\sample\per\second} and the external clock is working as intended.
Also the \ac{gandalf} firmware was adjusted to trigger on positive input signals instead of negative pulses.
Through a bug in the new firmware a surpassing of the maximum datarate of \SI{20}{\mega\byte\per\second} leads to incomplete events which cannot be decoded with the software of the \ac{gandalf}.
This should not be a problem for the application with the One Cell Prototype since the datarate from firt tests is much lower than the limit.

The most important settings of the \ac{emusic} boards were tested.
The first of these was the measurement of the input offset voltage used to adjust the overvoltage of the \acp{sipm} on a channel by channel level.
It was found, that the input \ac{dac}, which sets the offset only follows a linear behavior between \SI{100}{\dacu} and \SI{450}{\dacu}.
In this range the linear behavior measured with the \ac{emusic} board 2 is
\begin{align}
	V = asdlkjf
\end{align}
and for the \ac{emusic} board 6 it is
\begin{align}
	V = asdlkjf.
\end{align}
But the input offset differs between the different channels, therefore it is necessary to measure the input offset voltage while adjusting it.
For both boards the input \acp{dac} were set to get approximatly \SI{1}{\volt} input offset.

Also the pole-zero cancellation shaper was tested with different settings.
Therefore measurements with deactivated pole-zero cancellation shaper were made as a reference.
Also measurements with eight different settings for the shapers resistor at a constant setting for the shapers capacitor were done to see the effect of the resistor settings.
In order to see the effect of the capacitor eight measurements with different capacitor settings but the same resistor setting were performed.



Two dark count measurements were performed, one with and one without the pole-zero cancellation shaper.
For the measurement with the pole-zero cancellation shaper, the resistor setting 3 and the capacitor setting 31 were chosen.
The integral of the waveforms were histogramed.
For both measurements no single photoelectron peaks could be distinguished in the histogram. 
Hence it was not possible to use dark count measurements for calibration.


