\chapter{\ac{daq} performance at the \ac{desy} testbeam}
In october 2022 the One Cell Prototype was tested for one week at the \ac{desy} testbeam.
\autoref{fig:one_cell_testbeam} shows a picture of the the One Cell Prototype set up at the testbeam area.
The electron beam comes form the left, where it first passed a $2\times\SI{2}{\milli\meter\squared}$ square lead collimater.
Afterwards the electrons passed four plastic scintillators which are read out with \acp{pmt}.
A coincidence of the signals from all four \acp{pmt} is used to trigger the data aquisition.
After the scintitllators the One Cell Prototype is placed.
It stands on a rotary table to allow measurements with the beam coming from different angles, and on a \ac{desy} table, which enables left-right and up-down movement.
Measurements of each \num{10000} events were performed for XX positions by adjusting the position of the \ac{desy} table.
Each position was measured at different angles from \SIrange{0}{90}{\degree} in \SI{15}{\degree} steps by turning the detector on its rotary table.
For all positions and angles five measurements were done with variing beam energy starting from \SI{1.4}{\mega\electronvolt} up to \SI{5.4}{\mega\electronvolt} in \SI{1}{\mega\electronvolt} steps.






After the performance of the \ac{daq} was tested at the testbeam for one week with particles of known energy and known direction of movement, the long term performance of the one cell prototype and the \ac{daq} needs to be investigated.
For this purpose, the one cell prototype is assembled at the University of Freiburg, where it is supposed to be taking continuously data for a year.
Here the majority of events will be caused by cosmic muons.
By adding plastic scintillators with \acp{pmt} above and below the detector, trigger the data aquisition on a coincidence to only measure events, where the partice passed vertically through the detector.
Also a way to calibrate the detector needs to be found.
