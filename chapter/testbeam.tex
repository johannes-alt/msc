\chapter{\ac{daq} performance at the \ac{desy} testbeam}








After the performance of the \ac{daq} was tested at the testbeam for one week with particles of known energy and known direction of movement, the long term performance of the one cell prototype and the \ac{daq} needs to be investigated.
For this purpose, the one cell prototype is assembled at the University of Freiburg, where it is supposed to be taking continuously data for a year.
Here the majority of events will be caused by cosmic muons.
By adding plastic scintillators with \acp{pmt} above and below the detector, trigger the data aquisition on a coincidence to only measure events, where the partice passed vertically through the detector.
Also a way to calibrate the detector needs to be found.
