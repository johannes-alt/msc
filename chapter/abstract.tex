\begin{abstract}
    %The SHiP experiment is a proposed beam dump experiment to be build at the SPS which inteds to study physics beyond the Standard Model. It is planed as a zero background experiment and in order to fullfill this requirment, the Sourrunding Background Tagger (SBT) is needed. The SBT is a large scale structure consisting of around 2000 cell filled with the liquid scintillator LAB. In each cell two Wavelength-shifting Optical Modules collect the scintillation light and guide it to an array of SiPMs. The charge signals from the SiPMs can then be further amplified and digitized. During this thesis a readout for usage with a prototype of one of cells was assembled, characterized and tested for proper functionality. The readout consit of two main parts. One is a breakout board that can be plugged on the back of the SiPM PCB and which uses the eMUSIC chip as amplifier and shaper. The other are two GANDALF modules, which digitize the amplified output signals from the eMUSIC. In this thesis different settings of and options for the eMUSICs pole-zero cancellation shaper were investigated. Also the influence between high and low transimpedance of the eMUSIC were examined. For the use of two GANDALFs an external clock was assembled and tested and the GANDALF firmware was adjusted to enable the selftriggering on input signals with positive polarity. The eMUSIC boards were successfully used at the DESY testbeam and the long term performance of the full readout will be under investigation in the near futur.
    The SHiP experiment is a proposed beam dump experiment to be built at the SPS, which intends to study physics beyond the Standard Model.
    To meet SHiP's zero background requirement, the Sourrunding Background Tagger (SBT) encloses the hidden sector decay volume.
    The SBT is a large-scale structure consisting of around 2000 cells filled with the liquid scintillator LAB.
    In each cell, two Wavelength-shifting Optical Modules collect the scintillation light and guide it to an array of SiPMs.
    The charge signals from the SiPMs can then be further amplified and digitized.
    During this thesis, a readout for usage with a prototype of one of the cells was assembled, characterized, and tested for proper functionality.
    The readout consists of two main parts. One is a breakout board that can be plugged into the back of the SiPM PCB and uses the eMUSIC chip as an amplifier and shaper.
    The other is two GANDALF modules, which digitize the amplified output signals from the eMUSIC.
    This thesis investigated different settings and options for the eMUSICs pole-zero cancellation shaper.
    Also, the influence between high and low transimpedance of the eMUSIC was examined.
    For the use of two GANDALFs, an external clock was assembled and tested, and the GANDALF firmware was adjusted to enable the self-triggering of input signals with positive polarity.
    The eMUSIC boards were successfully used at the DESY test beam, and the long-term performance of the full readout will be investigated soon.
\end{abstract}
